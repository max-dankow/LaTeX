\documentclass[12pt]{article}
\usepackage[utf8]{inputenc}
\usepackage[T1]{fontenc}
\usepackage[russian]{babel}

\begin{document}
\tableofcontents
\newpage

\section{Модель вычислений}
\subsection{Постановка задачи и основные определения}


\newtheorem{Def}{Определение}
\newtheorem{Statement}{Утверждение}

Будем рассматривать следующую задачу.
Есть два участника процесса(два человека или два компьютера), которые должны совместно вычислить значение функции
$ f \colon X \times Y \rightarrow Z $, где $X, Y, Z$ некоторые конечные множества.
В качестве $Z$ обычно будем рассматривать множество значений бита \{0, 1\}.
Традиционно этих участников называют Алиса и Боб,
поэтому для удобства и соответствия литературе будем называть их так же.
Сложность их задачи состоит в том, что аргумент, на котором необходимо посчитать значение функции, разделен на две части:
у Алисы есть только $x \in X$, а у Боба только $y \in Y$.

Однако в их распоряжении есть некий абстрактный канал связи, через который они могут передавать друг другу данные. Передача по этому каналу связи может быть дорогой(или занимать значительное время), поэтому необходимо минимизировать количество битов, передаваемых в процессе вычисления функции.

Так же предполагается, что Алиса и Боб заранее знают функцию $f$ и договариваются о протоколе -
наборе соглашений о том, как и в каком порядке будет происходить обмен информацией.

\begin{Def}
Коммуникационным протоколом для вычисления некоторой функции
$ f \colon X \times Y \rightarrow Z $
называется ориентированное двоичное дерево, такое что:
\begin{enumerate}
    \item
    Каждой \textbf{листовой} вершине ставится в соответствие некоторый элемент их Z.
    \item
    Каждая \textbf{внутренняя} вершина помечена значением из $ \{ A, B \} $
    \item
    Для для каждой вершины $v_i$ с пометкой $A$ задана функция \linebreak
    $ g_i \colon X \rightarrow \{0, 1\} $
    \item
    Для для каждой вершины $v_j$ с пометкой $B$ задана функция \linebreak
    $ h_j \colon Y \rightarrow \{0, 1\} $
    \item
    Каждому ребру приписано значение из \{0, 1\}, а из каждой вершины, не являющегося листом, исходит ровно одно ребро с пометкой 0 и ровно одно с пометкой 1.
\end{enumerate}
\end{Def}

Выполнение протокола участниками вычисления начинается в корневой вершине.
На каждом шаге, переход осуществляется следующим образом. Пусть пометка очередной вершины $v_i$ равна $A$. Это означает, что сейчас Алиса должна применить функцию $g_i(x)$(соответствующую вершине $v_i$) к ее значению $x$. Если результат 0, то она отсылает Бобу значение 0 и переходит по ребру с меткой 0. Аналогично с 1.
Если же пометка $B$, то аналогично действовать должен Боб, применяя функцию $h_j(y)$ к его значению $y$.

Если текущая вершина это лист, то соответствующее ему значение $z \in Z$ объявляется результатом выполнения.
Это отражает идею, что к этому моменту все участники расчета знают этот ответ.

\begin{Def}
Протокол вычисляет функцию $ f \colon X \times Y \rightarrow Z $, \linebreak
если $ \forall x \in X \; \forall y \in Y $ при движении по графу протокола по описанным правилам исполнители попадут в лист, которому соответсвует $z=f(x,y)$.
\end{Def}

\subsection{Детерминированная коммуникационная сложность}

\begin{Def}
Сложностью коммуникационного протокола называется его глубина.
Детерминированной коммуникационной сложностью функции f называется минимальная сложность коммуникационного протокола, вычисляющего f. Обозначается CC(f).
\end{Def}

\begin{Statement}
Для функции $ f \colon X \times Y \rightarrow Z $ справедливы следующие тривиальные оценки:
    \begin{enumerate}
        \item
        $
            CC(f) \leq
            \lceil log|X| \rceil +
            \lceil log|Y| \rceil
        $
        \item
        $
            CC(f) \leq
            \lceil log|X| \rceil +
            \lceil log|Z| \rceil
        $
        \item
        $
            CC(f) \leq
            \lceil log|Y| \rceil +
            \lceil log|Z| \rceil
        $
        \item
        Если f сюръективна, то $ CC(f) \geq \lceil log|Z| \rceil $
    \end{enumerate}
\end{Statement}

Заметим, что в пункте 4 если $Z=\{0,1\}$, то $log_2|Z| = 1$, а значит в случае такого множества этот метод доказательства нижней оценки не эффективен.

\section{Нижние оценки}
\subsection{Одноцветные декартовы прямоугольники}

\begin{Statement}
Пусть $\Pi$ - некоторый коммуникационный протокол, а l - произвольный его лист.
Обозначим через $S_l$ множество таких пар (x, y) $\in X \times Y$,
что на входе (x, y) протокол $\Pi$ остановится в листе l.
Тогда $\exists A \subset X, B \subset Y$ такие что $S_l = A \times B$, а декартов прямоугольник $A \times B$ должен быть одноцветным с точки зрения значения функции f.
\end{Statement}

\begin{Def}
    $C^D(f)$ (англ. Disjoint Cover) - минимальное количество одноцветных прямоугольников, на которые можно разбить $X \times Y$.
    А величина $C^P(f)$ - минимально число листьев в протоколе.
\end{Def}

\begin{Statement}
    $$C^D \leq C^P \leq 2^{CC(f)}$$
\end{Statement}

\subsection{Метод трудных множества}



\end{document}

\documentclass[12pt]{article}
\usepackage[utf8]{inputenc}
\usepackage[T1]{fontenc}
\usepackage[russian]{babel}

\begin{document}
\tableofcontents
\newpage

\section{Модель вычислений}
\newtheorem{Def}{Определение}

Будем рассматривать следующую задачу.
Есть два участника процесса(два человека или два компьютера), которые должны совместно вычислить значение функции
$ f \colon X \times Y \rightarrow Z $, где $X, Y, Z$ некоторые конечные множества.
Традиционно этих участников называют Алиса и Боб,
поэтому для удобства и соответствия литературе будем называть их так же.
Сложность их задачи состоит в том, что аргумент, на котором необходимо посчитать значение функции, разделен на две части:
у Алисы есть только $x \in X$, а у Боба только $y \in Y$.

Однако в их распоряжении есть некий абстрактный канал связи, через который они могут передавать друг другу данные. Передача по этому каналу связи может быть дорогой(или занимать значительное время), поэтому необходимо минимизировать количество битов, передаваемых в процессе вычисления функции.

Так же предполагается, что Алиса и Боб заранее знают функцию $f$ и договариваются о протоколе -
наборе соглашений о том, как и в каком порядке будет происходить обмен информацией.

\begin{Def}

\end{Def}


\end{document}
